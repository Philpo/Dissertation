\chapter{Data Analysis and Evaluation}
\label{sec:eval}

This chapter will present and analyse the data gathered with the test plan detailed in the previous chapter. It will also evaluate the project hypotheses using the gathered data.

\section{Chosen Parameters}
The first stage of the data collection process was to choose suitable values for the parameters detailed previously.

\subsection{Mesh Size}
The maximum mesh size used was 300 by 300; this is the largest mesh size the can be run at real time frame rates with the most expensive integrator (RK4). This maximum was decreased by 50 in both dimensions, to a minimum size of 50 by 50, to give the six mesh sizes below.
\begin{itemize}
\item{300 by 300}
\item{250 by 250}
\item{200 by 200}
\item{150 by 150}
\item{100 by 100}
\item{50 by 50}
\end{itemize}

\subsection{Time Step}
The maximum time step was 20ms. This was the largest time step where at least one integrator was still stable for a 50 by 50 mesh. This values was decreased by 5ms giving the ranges of five time steps listed below.
\begin{itemize}
\item{20ms}
\item{15ms}
\item{10ms}
\item{5ms}
\item{1ms}
\end{itemize}

\subsection{Mass and Spring and Damping Coefficients}
These parameters were kept constant for every mesh size to avoid changing more than one variable between tests. Values were chosen that gave a reasonable cloth appearance. There are some visual issues, such as too much oscillation, with the values chosen at the larger mesh sizes but this is acceptable due to the focus of the project. The values used are as follows:
\begin{itemize}
\item{Mass = 100}
\item{Structural Stiffness = 20}
\item{Structural Damping = 7.5}
\item{Shear Stiffness = 20}
\item{Shear Damping = 7.5}
\item{Flexion Stiffness = 5}
\item{Flexion Damping = 2.5}
\end{itemize}
For Verlet, a constant damping factor of 0.5\% was used.

\section{Results}

\subsection{Explicit Euler}
Fig \ref{fig:ee fps sheet} shows the average simulation frame rate for every time step at every mesh size, in the sheet scenario. The graph shows two things, firstly that as mesh size increases average frame rate decreases and secondly that as time step increases average frame rate increases.
\\\\The data shows that varying mesh size can have a dramatic effect on the frame rate, this is shown well by Fig \ref{fig:ee mesh fps sheet}. This graph uses the average FPS for a 1ms time step, so that only one variable is changed. It has a strong negative correlation, which supports the conclusion that mesh size has a large effect on performance. Also, it clearly shows a dramatic frame rate decrease as mesh size is increased to 100 by 100 and 150 by 150; for a 100 by 100 mesh the decrease is approximately 1.4 times and for 150 by 150 approximately 5.4 times.
\\The decrease in frame rate can be explained by examining where the time each frame is spent. Fig \ref{fig:ee ft sheet} shows that the majority of the frame time is spent in the update function for a 300 by 300 mesh using a 1ms time step. This is the worst case test, but the conclusion is reflected across all other mesh sizes and time steps. By plotting the average time spent in the update function against mesh size, it is possible to explain why the frame rate decreases. Fig \ref{fig:ee mesh update sheet} shows this graph; as with Fig \ref{fig:ee mesh fps sheet} it shows data for a 1ms time step. This shows a strong positive correlation, thus it can be concluded that update time increases with mesh size. If the time for each update increases with mesh size, then the total time for each frame must increase also, thus reducing the frame rate of the simulation. 
\\Digging a little deeper, Fig \ref{fig:ee ut sheet} shows that calculating the internal forces, i.e. the spring forces, is the most expensive part of each update. By plotting this against mesh size, shown in Fig \ref{fig:ee mesh csf sheet}, a strong positive correlation is again observed. This is easy to explain. As mesh size increases, the number of particles increases, and therefore the number of springs also increases. More springs means that Equations \ref{eq:hooke equation} and \ref{eq:spring damping} must be calculated more, hence the time spent on internal forces increases.
\\\\The increase in update time with mesh size also explains the increase in frame rate as the time step increases. As the time step increases, the expensive update function is called less frequently, so overall frame time is decreased, thus increasing the frame rate. Fig \ref{fig:ee step fps sheet} shows the average frame rate plotted against the time step for a 300 by 300 mesh. This graph shows a positive correlation, thus supporting that frame rate increases with time step. It also shows that frame rate only increases once the time step exceeds some threshold; this is also shown in Fig \ref{fig:ee fps sheet}. Again, this result can be explained by examining the average update time. For the 300 by 300 mesh, the average update time was approximately 6ms, and frame rate only increased once the time step reaches 10ms. The reason for this should be obvious; both the 1ms and 5ms time steps are less than the total update time, therefore the update function will still be called every frame.
\\\\When looking at the results in the figures above, it is also important to look at the stability of each test. The stability of each test was evaluated subjectively, and the results are listed in Table \ref{tab:ee stability sheet}
\begin{table}[tp]
   \begin{minipage}{\textwidth}
      \begin{center}
         \begin{tabular}{c|c|c}
           Mesh Size & Time Step (ms) & Stable/Unstable\\
           \hline
           50 by 50 & 1 & Stable\\
           50 by 50 & 5 & Stable\\
           50 by 50 & 10 & Unstable\\
           50 by 50 & 15 & Unstable\\
           50 by 50 & 20 & Unstable\\
           100 by 100 & 1 & Stable\\
           100 by 100 & 5 & Unstable\\
           100 by 100 & 10 & Unstable\\
           100 by 100 & 15 & Unstable\\
           100 by 100 & 20 & Unstable\\           
           150 by 150 & 1 & Stable\\
           150 by 150 & 5 & Unstable\\
           150 by 150 & 10 & Unstable\\
           150 by 150 & 15 & Unstable\\
           150 by 150 & 20 & Unstable\\           
           200 by 200 & 1 & Stable\\
           200 by 200 & 5 & Unstable\\
           200 by 200 & 10 & Unstable\\
           200 by 200 & 15 & Unstable\\
           200 by 200 & 20 & Unstable\\           
           250 by 250 & 1 & Stable\\
           250 by 250 & 5 & Unstable\\
           250 by 250 & 10 & Unstable\\
           250 by 250 & 15 & Unstable\\
           250 by 250 & 20 & Unstable\\           
           300 by 300 & 1 & Stable\\
           300 by 300 & 5 & Unstable\\
           300 by 300 & 10 & Unstable\\
           300 by 300 & 15 & Unstable\\
           300 by 300 & 20 & Unstable\\
         \end{tabular}
      \end{center}
   \end{minipage}
   \caption{Stability results for explicit Euler (sheet)}
   \label{tab:ee stability sheet}
\end{table}
As can be seen, for the sheet scenario, explicit Euler was really only stable with a 1ms time step, with the exception of the 50 by 50 mesh, where it was stable with a 5ms time step as well.
\\\\The data for the flag scenario shows similar correlations to the sheet scenario. Figs \ref{fig:ee fps flag} and \ref{fig:ee mesh fps flag} show that as mesh size increases FPS decreases, with the same initial dramatic decreases as the sheet scenario; the decrease for a 100 by 100 mesh is approximately 2.3 times and approximately 5 times for 150 by 150. As with the sheet scenario, the majority of the frame time is still spent within update (see Fig \ref{fig:ee ft flag}) and Fig \ref{fig:ee mesh update flag} displays a strong positive correlation as well.
\\Fig \ref{fig:ee ut flag} shows a slightly different time breakdown within the update function. Calculating the internal forces is still the most expensive part, but the cost of calculating external forces has risen significantly over the sheet scenario; an approximate increase of 6.7 times. This is because the flag scenario includes wind as an additional external force. In order to apply wind, the particles must be split into triangles and the surface normal of every triangle calculated at every time step.
\\Figs \ref{fig:ee fps flag} and \ref{fig:ee step fps flag} also show that as time step increases the FPS increases with it. They also support the conclusion that frame rate only increases once the time step exceeds the total update time; again the frame rate for the 300 by 300 mesh only increased at 10ms, because the average update time was approximately 8ms.
\\The stability of the flag scenario tests, listed in Table \ref{tab:ee stability flag} are also similar to the sheet scenario, with the only difference being that a 5ms time step and 100 by 100 mesh was stable for the flag scenario.

\begin{table}[tp]
   \begin{minipage}{\textwidth}
      \begin{center}
         \begin{tabular}{c|c|c}
           Mesh Size & Time Step (ms) & Stable/Unstable\\
           \hline
           50 by 50 & 1 & Stable\\
           50 by 50 & 5 & Stable\\
           50 by 50 & 10 & Unstable\\
           50 by 50 & 15 & Unstable\\
           50 by 50 & 20 & Unstable\\
           100 by 100 & 1 & Stable\\
           100 by 100 & 5 & Stable\\
           100 by 100 & 10 & Unstable\\
           100 by 100 & 15 & Unstable\\
           100 by 100 & 20 & Unstable\\           
           150 by 150 & 1 & Stable\\
           150 by 150 & 5 & Unstable\\
           150 by 150 & 10 & Unstable\\
           150 by 150 & 15 & Unstable\\
           150 by 150 & 20 & Unstable\\           
           200 by 200 & 1 & Stable\\
           200 by 200 & 5 & Unstable\\
           200 by 200 & 10 & Unstable\\
           200 by 200 & 15 & Unstable\\
           200 by 200 & 20 & Unstable\\           
           250 by 250 & 1 & Stable\\
           250 by 250 & 5 & Unstable\\
           250 by 250 & 10 & Unstable\\
           250 by 250 & 15 & Unstable\\
           250 by 250 & 20 & Unstable\\           
           300 by 300 & 1 & Stable\\
           300 by 300 & 5 & Unstable\\
           300 by 300 & 10 & Unstable\\
           300 by 300 & 15 & Unstable\\
           300 by 300 & 20 & Unstable\\
         \end{tabular}
      \end{center}
   \end{minipage}
   \caption{Stability results for explicit Euler (flag)}
   \label{tab:ee stability flag}
\end{table}

\subsection{Verlet}

\subsection{Midpoint}

\subsection{Fourth Order Runge-Kutta}

\section{Evaluation}

\subsection{Null Hypothesis}

\subsection{Alternative Hypothesis}

