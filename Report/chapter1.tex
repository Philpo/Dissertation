\chapter{Introduction}

The visual fidelity of video games has increased dramatically in recent years largely due to developments in discrete graphics hardware, or graphics processing units (GPUs). As a result, there is a need for realistic cloth in order to sell the appearance of video game characters and scenes. It is essential that the cloth should react to dynamic behaviour, such as a character moving or the wind blowing, and it must be animated in real time, i.e. at a minimum of 30 frames per second (FPS). These requirements render many cloth simulation techniques inappropriate, as they are too computationally expensive to be run in real time. The Mass-Spring model is one technique that is appropriate for use in video games, and is the most popular method for handling cloth for games. Whilst not necessarily providing 100\% accuracy, Mass-Spring models result in real time frame rates and give visually pleasing animation, which is good enough for games.
\\Mass-Spring models use forces, calculated using Newtonian mechanics, to animate the cloth. This results in a series of differential equations that must be approximated by a numerical integrator at discrete time intervals. There are many different integrators that can be used, the choice of which can directly affect the performance of the simulation. Some integrators must necessarily use small time intervals in order to maintain the stability of the cloth, and this increases the frequency of the integration calculations thus reducing performance.
\\This project will investigate the effect different integration methods have on a real time cloth simulation using the Mass-Spring model.

\section{Project Aims}
The aim of this project is to investigate the performance effects of different integration methods on real time cloth simulation using the Mass-Spring model.

Based on this aim, the objectives of the project are as follows:
\begin{itemize}
\item {To research cloth simulation and numerical integration techniques }
\item {To implement the Mass-Spring model for cloth simulation and several integrators, including explicit Euler}
\item {To investigate the performance effects of the implemented integration methods on the simulation}
\end{itemize}

\section{Hypotheses}
\textcite{Volino2001} were the first to investigate the performance impact of different integrators on cloth simulation. Their results, however, are extremely outdated, due to the vast increase in CPU power since 2001; from a 200MHz workstation CPU to the 3GHz+ CPUs available in modern desktops and laptops. Later work by \textcite{Wang2009a} also investigated the impact of different integrators, and found that integrator choice still has an impact on simulation performance. However they make no reference to the hardware on which their experiments were run.
\\As a result, two hypotheses are proposed for this project, a null and an alternative.

\subsection{Null Hypothesis}
The null hypothesis is that all integration methods result in a real time cloth simulation when running on modern hardware.
\\This was chosen because \textcite{Wang2009a} make no reference to the hardware used, and the simulation developed will be run on an Intel I7 4770K at 4.2GHz, so it may be the case that there are no performance concerns with this modern hardware.

\subsection{Alternative Hypothesis}
The alternate hypothesis is that some integration methods are prohibitively expensive for real time cloth simulation and other methods provide better performance.
\\Some researchers, such as \textcite{Zeller2005} and \textcite{Tang2013}, have proposed GPU accelerated approaches to Mass-Spring models which suggests that performance is still a consideration. Third party physics engines, such as NVIDIA \textsuperscript{\textregistered} PhysX\textsuperscript{\textregistered}, also use GPU accelerated models \parencite{Kim2011} again suggesting performance of Mass-Spring models are still a concern.

\section{Report Structure}
This remainder of this report will be structured as follows:
\begin{itemize}
  \item{Chapter 2: literature review. An overview of cloth simulation techniques will be given and the Mass-Spring model and some numerical integrators described in detail}
  \item{Chapter 3: design and implementation. This chapter will describe the development methodology, design and some of the implementation details of the project}
  \item{Chapter 4: test plan. Details on how the hypotheses will be tested are described}
  \item{Chapter 5: data analysis and evaluation. The data gathered from the testing process is presented and analysed, and the hypotheses evaluated}
  \item{Chapter 6: conclusion. The results of the project are summarised and some suggestions for future work proposed}
\end{itemize}
