\chapter{Conclusion}

This project aimed to investigate how well stochastic optimisation techniques could learn to play Connect 4. Specifically, three population based computational intelligence techniques were investigated; the genetic, evolution strategies and particle swarm optimisation algorithms.
\\After the initial research, the development of the project was split into two phases; firstly to implement the chosen algorithms into a generic optimisation library and secondly to adapt this library to play Connect 4 in The Arena. This allows the objectives of the project to be evaluated separately.
\\The first developmental stage aimed to achieve the following project objective: "Implement and test the chosen population based optimisation algorithms, producing a generic implementation". The extent to which this objective was met was evaluated by testing the performance of each algorithm in maximising and minimising De Jong's function 1. The test results for this phase show that this objective has been strongly achieved; efficient and accurate implementations of the chosen algorithms have been produced, able to maximise and minimise with a number of selection and recombination modes and the design of the system allows new functions and algorithms to be added with minimal effort. The only limitation to the success of phase 1 is the fact that a small number of the selection and recombination modes have either undefined or poor performance. 
\\The objectives of the second phase of development were to "Adapt the generic implementation to work with Connect 4 in The Arena" and then to "Evaluate the effectiveness of the different algorithms in playing Connect 4". The evaluation of the algorithms' effectiveness would be tested by playing a number of games against both the author and the recursive player module provided by The Arena. The adaptation to The Arena was achieved successfully, with the algorithms able to play Connect 4 and perform the necessary optimisation steps using Connect 4 as a fitness function. The adaptation of the algorithms to The Arena took longer that anticipated however, and therefore the development of an effective playing algorithm was limited, because of the limited time frame this project had for development. As such, the effectiveness of the project was expected to be hindered. 
\\The test results show that only one of the algorithms, evolution strategies, was able to effectively learn to play Connect 4, when playing against the author, and even then only when playing as player1. None of the algorithms were able to closely match the recursive player, even at low time limits. This may show that the algorithms are not appropriate for playing Connect 4, or it may be that the design of the playing algorithm or the configuration parameters for the algorithms were limiting factors. More rigorous testing would be needed to produce more conclusive results.
\\Throughout the testing of the two developmental phases, the performance of the algorithms have been compared to meet the final objective of the project: "Compare the effectiveness and efficiency of the different algorithms". During phase 1 testing, the accuracy (effectiveness) and convergence speed (efficiency) of the algorithms were compared. It was found that evolution strategies was the most effective and efficient algorithm; it was able to find the maximum within 350 generations and its accuracy was limited only by the number of generations. Particle swarm optimisation was the least efficient algorithm, taken 900 generations to find the maximum, but considering the operation of the algorithm, this was expected. It was however, the second most accurate algorithm, again limited only by the number of generations. The genetic algorithm was shown to be the least accurate, only able to able find a minimum of $5.00997\e{-4}$, however this minimum was limited by the number of bits used to represent each gene, so if more bits were used smaller minima could be found. It was also the second most efficient algorithm, taking only slightly longer than evolution strategies to optimise the function. During phase two testing, the effectiveness of each algorithm at playing Connect 4 was compared, and again evolution strategies was shown to be the most effective, actually capable of beating the author and showing some clever behaviour in setting traps to ensure it won.
\\Overall, this project has achieved mixed results. On the one hand, the objective of developing a generic optimisation library using population based techniques has been almost wholeheartedly met and this library has been adapted to allow optimisation of rules for playing Connect 4. The results of this optimisation though have been poor, with the algorithms mostly unable to play Connect 4 with any real effectiveness. Whether this poor performance is because of the time constraints of the project limiting the training period and design of the playing algorithm, or that the system design or algorithms themselves are inappropriate for Connect 4 is unknown and would need more rigorous testing to establish. If the focus of the project had been more towards the Connect 4 side of the problem, and not the implementation of the algorithms themselves, better results may have been achieved.

\section{Future Work}
Because of the mixed results obtained by the project, there is scope for future research.
\\\\Firstly, the system as it currently is should be more rigorously tested, with longer training periods and different parameter setups. This would allow the design of the system to be more effectively evaluated and establish whether it needs to be redesigned.
\\\\Now the algorithms have been adapted to optimise in The Arena, more time should be spent designing an effective playing algorithm and genotype for the individuals. This should be rigorously tested, and then a more conclusive answers as to whether the chosen algorithms are appropriate for playing Connect 4 could be obtained. The optimisation library could be adapted as well, to include more selection and recombination modes, which could improve the optimisation performance of the algorithms and they may play better as a result.
\\\\Other computational intelligence techniques could be developed to play Connect 4, in order to show whether computational intelligence is something that can be effectively applied to Connect 4 at all.

\section{Closing Remarks}
This project, whilst achieving slightly disappointing results, has been an interesting undertaking. Perhaps, if a computational intelligence library had been used, instead of implementing the algorithms manually, then better results could have been achieved, as more time could have been spent design the playing algorithm and genotype. With hindsight, perhaps this path should have been taken but one of the author's main aims when suggesting this project was to gain practical experience implementing computational intelligence techniques, and in that respect this project has been very successful.